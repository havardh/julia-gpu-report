\begin{markdown}

\chapter{Background}
  
Programming a modern \gls{GPU} is a difficult problem. A lot of
architecture specific aspects has to be taken into
consideration. Traditionally, these devices can only be programmed in
low level languages such as C/C++. Software libraries and language
bindings does exist for higher level languages, enabling a user of
such a language to make use of the device. However, the
kernel\footnote{A kernel is the piece of code that is executed in
  parallel, this is usually implemented as a function.} implementation
is still in a low level language.

The Julia programming language aims to unify the two worlds of high
productivity, given by abstractions, and high performance, given by a
statically compiled language. This will enable users of this seemingly
high level language to dive into optimizing their applications, without
having to resort to a lower level language for the
implementation.

The PTX.jl library presented in this report extends the effort made by
the Julia language, to the realm of \gls{GPU} programming. This is
possible due to the fact that Julia is implemented on top of
\gls{LLVM} \cite{llvm}, which also is the basis for both NVVM
\cite{nvvm} and SPIR \cite{spir}, the intermediate representations of
both the NVIDIA and the \gls{OpenCL} compiler.

# History and State of the art #

The first programming of a \gls{GPU} started out with a fixed pipeline
with programmable steps used for rendering 3D graphics. In 2006, with
the introduction of __\gls{CUDA}__ \cite{cuda} by NVIDIA and
__\gls{APP}__ \cite{app} by AMD in 2007, \gls{GPGPU} was
introduced. \gls{GPGPU} entailes that the pipeline is fully
programmable. This enables the \gls{GPU} to be used for non-graphic
related applicaion.  Microsoft released __DirectCompute__ along with
DirectX11 \cite{directx} in 2008 that exposes a \gls{API} for the
\gls{GPU} to the programmer in C++. In 2009 the Khronos Group released
the first specification of __\gls{OpenCL}__ \cite{opencl} which tries
to provide a platform independent language for programming
heterogeneous systems. The __OpenACC__ \cite{openacc} standard was
released in 2011, enabling programmers to execute code on \gls{GPU}s
with directives as OpenMP \cite{openmp} enables multiple threads. In
2012 Microsoft released __C++ \gls{AMP}__ \cite{c++amp}, a C++
language- and programming model extension. Microsoft released a
compiler for their own DirectCompute framework, and Intel and AMD has
later released compilers based on the specification \cite{shelvinpark}
\cite{amd-amp}. __Numba__ was \cite{numba} introduced by Continuum
Analytics and is still in development (2014), the compiler enables
executing Python code on a NVIDIA GPU and is built on \gls{LLVM}.

Out of these \gls{CUDA} and \gls{OpenCL} are most suited for
considerations in this projects, as they can be used as compiler
targets. The others mentioned are software libraries that build on
these underlying standards. The Numba compiler comes closest in
functionalty of the approach taken by PTX.jl.

# The SIMT model #
\label{sec:simt}

This section will provide a brief introduction to the programming
model exposed to the programmer for both OpenCL \footnote{In general,
  OpenCL exposes a model called \gls{SPMD} \cite{opencl}, but on a GPU
  with an underlying vector architecture this can be considered as
  SIMT.}  and CUDA. \gls{SIMT} is an abstraction over the hardware
technique of \gls{SIMD}. The model provides the programmer with the
abstraction that a kernel is executed in an independent thread. The
kernel usually takes large one-dimentional arrays as input and output
parameter. The environment defines index functions that the programmer
can use to access elements of the data structures. An example kernel
for vector addition is given in Figure \ref{smit:add}.

\begin{figure}[H]
  \begin{minted}[linenos,numbersep=5pt,gobble=2,frame=lines,framesep=2mm]{c}
  __kernel void add(long *a, long *b, long *c) {
    int i = get_global_id(0);
    c[i] = a[i] + b[i];
  }
  \end{minted}
  \caption{OpenCL Vector addition kernel}
  \label{smit:add}
\end{figure}

On line 2, the kernel finds its index in the first dimension, and on
line 3 a single element addition is performed. The _Host_ (\gls{CPU})
schedules the kernel to be executed on the same number of threads as
the length of the array. The hardware on the _Device_ (\gls{GPU})
ensures that the index function returns the correct index for each
thread.

# Intermediate Representations #

In the last few years both \gls{OpenCL} and \gls{CUDA} has started
targeting the \gls{LLVM} compiler infrastructure. NVIDIA released part
of their compiler in 2009 as open source. This included the \gls{PTX}
backend \cite{nvptx} and the libNVVM \cite{libnvvm} optimizing
compiler, both based on \gls{LLVM}. They also introduced a new
intermediate representation called NVVM, which is largely based on
\gls{LLVM} \gls{IR}. In 2012 the Khronos Group introduced \gls{OpenCL}
\gls{SPIR}, an intermediate representation also based on \gls{LLVM}
\gls{IR}. As both these \glspl{IR} was considered as targets for the
PTX.jl library, a short introduction is given in the following
subsections.

## LLVM IR ##

This section will briefly introduce the \gls{LLVM} intermediate
representation. The \gls{SPIR} and \gls{NVVM} representations
described in the next two subsections are extensions to this
representation.

\gls{LLVM} \gls{IR} is a compiler internal intermediate
representation. It is intended as a good representation for compiler
optimization and analysis. The representation has three forms; a human
readable format (.ll), a bitcode format (.bc) and an in-memory
format. These three forms are all equivalent, therefore the human
readable format is presented here.

The format is on \gls{SSA} form, which makes reasoning about the code
easy. The language consists of modules that contains a _target_,
_function declarations_ and _implementations_, _type specifications_,
_attributes_ and _metadata nodes_.

Figure \ref{fig:llvm} lists a typical \gls{LLVM} module.

\begin{figure}[H]
  \begin{minted}[linenos,numbersep=5pt,gobble=2,frame=lines,framesep=2mm]{llvm}
  ; ModuleID = 'add.bc'
  target datalayout = "e-m:o-i64:64-f80:128-n8:16:32:64-S128"
  target triple = "x86_64-apple-macosx10.9.0"
  
  ; Function Attrs: nounwind ssp uwtable
  define void @add(i64* %a, i64* %b, i64* %c) #0 {
    ; code 
    ret void
  }
  
  declare i64 @get_global_id(i32)
  
  attributes #0 = { nounwind ssp uwtable  }
  
  !llvm.ident = !{!0}
  
  !0 = metadata !{metadata !"Apple LLVM version 6.0 (clang-600.0.54) (based on LLVM 3.5svn)"}
  \end{minted}
  \caption{A module with a function in LLVM IR}
  \label{fig:llvm}
\end{figure}


Lines 1 and 2 in Figure \ref{fig:llvm} specifies the underlying target
architecture with the _datalayout_ and the _target triple_. Next on
line 6 is a _function implementation_ of the _add_ function, the #0 on
the end of line binds the function to the attributes _nounwind ssp
uwtable_ specified on line 13. Above the attribute on line 11 is a _
function declaration_ of _get_global_id_. The last two lines 15 and 17
specifies metadata. LLVM contains both named metadata and unnamed. The
unnamed section gets assigned numbers !0, in this case on line 17.  The
_!llvm.ident_ on line 15 referes to this unnamed node. 

## SPIR ##

\gls{SPIR} is an \glspl{OpenCL} format for representing kernels for
execution on heterogeneous systems. The specification is a mapping
from OpenCL C into LLVM IR. SPIR does not add a lot of extra to the
LLVM IR. The notable differences are new _target datalayout_ and
_triple_, a _named metadata node_ for specifying kernels,
specification of _address space qualifiers_ and _name mangling
rules_. In addition, the kernel arguments must be included in the
metadata section. Figure \ref{fig:spir} lists a \gls{SPIR} module
containing the kernel definition for the _add_ function.


\begin{figure}[H]
  \begin{minted}[linenos,numbersep=5pt,gobble=2,frame=lines,framesep=2mm]{llvm}
  ; ModuleID = 'OpenCL Module'
  target datalayout = "e-p:64:64:64- ... -v1024:1024:1024" ; Not complete
  target triple = "spir64-unknown-unknown"
  
  declare i64 @_Z13get_global_idj(i32)
  
  define void @add(i64 addrspace(1)*, i64 addrspace(1)*, i64 addrspace(1)*) {
    ; code
    ret void
  }
  
  !llvm.ident = !{!0}
  !opencl.kernels = !{!1}
  
  !0 = metadata !{metadata !"Apple LLVM version 6.0 (clang-600.0.54) (based on LLVM 3.5  svn)"}
  !1 = metadata !{void (i64 addrspace(1)*, i64 addrspace(1)*, i64 addrspace(1)*)* @add}
  \end{minted}
  \caption{A module with a kernel definition in SPIR}
  \label{fig:spir}
\end{figure}

### The kernels metadata node ###

All SPIR modules must contain the named metadata node
__opencl.kernels__. The value of this node is a list of the kernel
functions in the module. For each kernel the signature must be
included in the metadata section. As seen in Figure \ref{fig:spir},
the opencl.kernels node points to the method signature in _!1_ and
this node matches the signature of the function.

### Address Space Qualifiers ###

The qualifier used in the parameter list of the add functions, shows
how address space qualifiers are utilized in SPIR. This qualifiers
denotes what type of memory is used on the device.

\begin{table}[H]
  \centering
  \begin{tabular}{|l|l|l|}
    \hline
    Qualifier & Memory   & Comment \\
    \hline
    \hline
    0         & private  & Private to each work item \\
    \hline
    1         & global   & Global for the entire device \\
    \hline
    2         & constant & Global for the entire device \\
    \hline
    3         & local    & Private to each work group \\
    \hline
    4         & generic  &  \\
    \hline
  \end{tabular}
  \caption{Address Space Qualifiers for SPIR}
  \label{tab:spir:addr}
\end{table}
  
The example in Figure \ref{fig:spir} uses only the global memory
space.

### Name Mangling ###

OpenCL contains many built-in overloaded functions, like the math
function __sin__. Name mangling is used to distinguish between the
different implementations based on their argument types. In Figure
\ref{fig:spir} on line 5 the _get_global_id_ is mangled into
__Z13get_global_idj_, the mangling rules are based on the rules for
the Intel Itanium.

## NVVM ##

\gls{NVVM} is NVIDIAs intermediate representation, used in their
\gls{LLVM} based compiler for \gls{CUDA}. Like \gls{SPIR}, it
represents device kernels that are intended to be executed on a
device, but solely \glspl{GPU}. \gls{NVVM} adds to \gls{LLVM} \gls{IR}
a set of intrinsic functions for controlling the \gls{GPU}. Among
these are barriers, address space conversions and special register
accessors. The \gls{NVVM} Cuda Compiler compiles \gls{NVVM} \gls{IR}
into \gls{PTX}, which is loaded into the devices with either the
\gls{CUDA} or \gls{OpenCL} driver provided by NVIDIA. Like \gls{SPIR},
it defines targets, address space qualifiers and a metadata node to
declare a function as a kernel.

\begin{figure}[H]
  \begin{minted}[linenos,numbersep=5pt,gobble=2,frame=lines,framesep=2mm]{llvm}
  ; ModuleID = 'OpenCL Module'
  target datalayout = "e-p:32:32:32- ... -v1024:1024:1024" ; Not complete
  target triple = "nvptx64-nvidia-cuda"
  
  declare i32 @llvm.nvvm.read.ptx.sreg.tid.x()
  
  define void @add(i64 addrspace(1)*, i64 addrspace(1)*, i64 addrspace(1)*) {
    ; code
    ret void
  }
  
  !llvm.ident = !{!0}
  !nvvm.annotations = !{!1}
  
  !0 = metadata !{metadata !"Apple LLVM version 6.0 (clang-600.0.54) (based on LLVM 3.5svn)"}
  !1 = metadata !{void (i64 addrspace(1)*, i64 addrspace(1)*, i64 addrspace(1)*)* @add,
  metadata !"kernel", i32 1}
  \end{minted}
  \caption{A module with a kernel definition in NVVM}
  \label{fig:nvvm}
\end{figure}

### NVVM Annotations ###

The metadata node __nvvm.annotations__ explicitly marks the node as a
kernel node on line 17 in figure \ref{fig:nvvm}. This is done as the
annontations is also used for global variables etc. These are not used
in the PTX.jl library.

### Built-in Intrinsics ###

\gls{NVVM} comes with a set of built in intrisics functions. The
definition of the _threadIdx.x_ variable, accessible inside \gls{CUDA}
kernels, is implemented as such and intrinsic. Its definition is given
at line 5 of Figure \ref{fig:nvvm}.

### Address Space Qualifiers ###

Like \gls{SPIR}, \gls{NVVM} specifies address space qualifiers. They
differ only slightly from the OpenCL definition and is more tailored
to the NVIDIA GPU architecture. Table \ref{tab:nvvm:addr} tabulates
the different qualifiers and their usage.

\begin{table}[H]
  \centering
  \begin{tabular}{|l|l|l|l|}
    \hline
    Qualifier & Memory   & Comment                      & OpenCL equivalent \\
    \hline
0         & generic  &                   & generic  \\
1         & global   & Global for the entire device & global \\
3         & shared   & Private to each work group   & local \\
4         & constant & Global for the entire device & constant \\
5         & local    & Private to each work item    & private \\
    \hline
  \end{tabular}
  \caption{Address Space Qualifiers for NVVM}
  \label{tab:nvvm:addr}
\end{table}


## Summary ## 

As the above discussion shows, SPIR and NVVM are in the gist very
similar. They both define multiple address spaces, built-in
functionality and a way to distingush kernel functions. The largest
incompability is the definition of the address qualifiers. 

## Support ##

Both these intermediate representations are inteded as compiler
targets for new programming languages. The NVVM is only supported by
NVIDIAs drivers and can only be executed on NVIDIA GPUs. SPIR, on the
other hand, can potentialy execute on all platforms that execute
OpenCL. Unfortunalty, at the time of writing an implementation on
NVIDIA is not available, but both Intel an AMD has support for SPIR in
their newest hardware.

# A governing factor in Hardware Platform #

An external governing factor for the PTX.jl implementation has been the
available hardware for testing. The workstation listed in Table
\ref{tab:hw} was provided in the NTNU
HPC-Lab\footnote{http://research.idi.ntnu.no/hpc-lab/} by Dr. Anne
C. Elster.


\begin{table}[H]
  \centering
  \begin{tabular}{|l|l|l|}
    \hline
    & CPU & GPU \\
    \hline
    \hline
    Name & Intel i7 & NVIDIA GTX 480 \\
    \hline
    \# of Cores & 4 & 448 \\
    \hline
    Memory & 4GB & 1536MB \\
    \hline
  \end{tabular}
  \caption{Hardware platform}
  \label{tab:hw}
\end{table}

As this platform consists of a NVIDIA GPU, the IR chosen was NVVM. As
the outlines in this section implies, changing the library to target
SPIR instead can be easily done.

\end{markdown}
