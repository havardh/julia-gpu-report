\begin{abstract}
The race for better graphics performance has led to sophisticated
processor designs targeting the parallel behavior of computer
graphics. These processing units, called \gls{GPU}, have been subject
to a lot of funding, and therefore research causing an accelerated
development of high performance specialized systems. This, accompanied
by the development of software components like CUDA and OpenCL, have
exposed these systems to the \gls{HPC} community through so-called
\gls{GPGPU} programming. These libraries exposes low level APIs and
require the applications to be designed in languages like C/C++.

In the realm of technical computing, a lot of applications are
prototyped in high-level languages like Python, Matlab, R and
Mathematica. These languages enables high productivity, but suffers
from low performance due to poor interpreting/compilation
processes. The prototypes must then be re-implemented in languages that
target performance, but not productivity. Following the introduction of
the \gls{LLVM} compiler infrastructure, new languages that target both
these concerns have been made possible.

This technical report explores the state of the art in joining these
two efforts. It describes \textit{PTX.jl}\footnote{https://github.com/havardh/PTX.jl, Commit: 5a1cf2b2dbe02fdd3f69f85a10dc24b18e18b89a}, a library to
make the \gls{GPU} programmable from a high level language. The
language chosen is the Julia programming language that targets
technical computing, built on the \gls{LLVM} infrastructure. The
language has the same emphasis on making high performance while
maintaining high productivity, as the PTX.jl library.
\end{abstract}

\glsresetall
