\chapter{Future Work}

The PTX.jl library main design objective was to be a full system
integration. This leads to the possibility to optimize and extend each
sub-component of the system. Here includes a more robust compiler (see
Section \ref{sec:disc:comp}) and implementing all of primitive types
in Julia. In addition the alternative paths mentioned in Chapter
\ref{chap:meth} such as a full fledged Julia language implementation on
the GPU can be considered.

From a performance viewpoint the fact that PTX.jl only supports global
addressing is a key limitation. To be able to optimize kernels for
speed, extensive utilization of local memory is required. The OpenCL
runtime library includes a lot of functions operating on
vectors. Adding vector types \footnote{\textit{Vector types} here referes to
  OpenCLs \textit{float4} and friends.} would enable utilizing these
functions.
