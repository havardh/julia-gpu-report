\chapter{Discussion}
\begin{markdown}

The contents of this chapter discusses the design and implementation
of the PTX.jl described in Chapters \ref{chap:impl} and
\ref{chap:meth}, and the results presented in Chapter \ref{chap:res}.

# Design #

## PTX.jl as a library ##

As mentioned in Section \ref{sec:meth:lib-b-comp}, the decision to
implement PTX.jl as a library was not the most obvious one. This was
not a clear decision taken up front, but more an alternative that
materialized when the changes to compiler was removed due to library
level alternatives. The most important parts was the fact that
inlining could be disabled and implementing multiple data types
primitive data types was made possible. The library can serve as a
basis to future investigate new patterns for GPU programming in Julia
without having modify the compiler. It also provides a small codebase
to experiment with writing compiler passes for the IR generated.

## Unboxing of arrays ##
 
The approach mentioned in Section \ref{sec:meth:arrays:low-level}, the
unboxing of arrays, was considered an easy first iteration step in
order to establish the whole prototype system. This prototype can now
be extended with more sophisticated translation from Julia to PTX
code. The performance results depicted in Section \ref{chap:res} can
serve as a benchmark for future higher level implementations. 

# Implementation #

## Fragile kernel compiler ##
\label{sec:disc:comp}
The kernel compiler in PTX.jl is rather fragile. Introducing to much
complexity in the Julia kernel function will cause either the _module
generator_ or the _lowering compiler_ to fail in a non-user-friendly
fashion.

# Result #

## How is PTX.jl holding up ##

We see from both benchmarks presented in Chapter \ref{chap:res} that
the performance of the kernel produced by PTX.jl is comparable to both
OpenCL C and CUDA C. This indicates that the method of lowering
incurs no overhead over the kernels developed in a lower level
language. 

## Offloading work to the GPU ##

The benchmark presented in the Section \ref{sec:res:mm} indicates that
offloading even relatively small matrix computations to the GPU will
provide a speedup. This holds true even, when the implementations
presented in this reported are unoptimized for performance. Usually
when looking to exploit the parallelizm of a GPU, the entire
application is implemented on the GPU. Given these results, there is
reason to explore a more fine-grained parallelization of a program. A
proposition for a framework for this is presented in the next
paragraph.

### Framework for fine-grained offloading ###

Building on the _GPUArray_ type presented in Section
\ref{sec:meth:ptx}, a framework for fine-grained offloading can be
built. This can be done by extending the type with operators and
methods for Matrix and or Vector operations. The operations can also
have built in auto-tuning, to decide when to offload to the \gls{GPU}
and when to keep the computation on the \gls{CPU}.

## Integer vs. Floating point performance in Matrix Multiplication ##

Summarizing the results in Section \ref{sec:res:mm}, we see that the
floating point performance of the \gls{CPU} is far better than the
integer. The degradation of integer multiply performance on the
\gls{CPU} enables a speedup with a naive GPU implementation already at
$38*38$ matrices for 64-bit integers values. This difference seems to
come from the fact that matrix multiplication scales badly on the CPU,
and the GPU implementation scales equally well on both the floating
point and the integer benchmarks.


\begin{figure}[H]
  

  \begin{minted}[linenos,numbersep=5pt,gobble=2,frame=lines,framesep=2mm]{julia}
  julia> A = rand(Int64, 2^10, 2^10);
  julia> B = rand(Int64, 2^10, 2^10);
  julia> A * B;
  julia> gc()
  julia> @time A * B;
  elapsed time: 1.354795104 seconds (8402840 bytes allocated)
  
  julia> A = rand(Float64, 2^10, 2^10);
  julia> B = rand(Float64, 2^10, 2^10);
  julia> A * B;
  julia> gc()
  julia> @time A * B;
  elapsed time: 0.070830169 seconds (8388800 bytes allocated)
  \end{minted}
  \caption{Comparing Floating point and integer performace}
  \label{fig:julia:mm}
\end{figure}

Figure \ref{fig:julia:mm} shows a minimal benchmark displaying the
performance difference for $1024*1024$ matrices. We see that the
floating point version is 19x faster than the integer version.


## Notes on bias when timing a dynamic programming language ##

When timing benchmarks; a lot of parameters might effect the
results. Special for the timings of the results presented here, are
mainly the three factors \gls{GC}, \gls{JIT} and Asynchronous
Dispatch.

### Garbage Collection ###

In an automatically garbage collected language, this mechanism must be
assumed to possibly trigger at any time. There exists different types
of collection algorithms and they will affect the results in different
ways. In the Julia language we have two mechanism to remove this
biased from our results. Firstly the timing macro \textit{@time}
reports the percentage of time spend in the \gls{GC}. By ensuring that
this number is zero we eliminate the overhead. This can be ensured by
the other mechanism, namely the \textit{gc()} function. This function will
run a blocking garbage collection iteration. By inserting this before
the timed function call, we can increase likelihood that the \gls{GC}
will not be triggered while timing. This is unless the function call
it self allocates and deallocates enough memory to be subject to
\gls{GC}. But this will show up in the report from the \textit{@time}
function.


\begin{figure}[H]
  \begin{minted}[linenos,numbersep=5pt,gobble=2,frame=lines,framesep=2mm]{julia}
  gc()
  @time foo()    
  \end{minted}
  \caption{Running Garbage Collection proir to timing}
\end{figure}

### Just-In-Time compiling ###

The Julia programming language is implemented with a \gls{JIT}. This
means that the functions are compiled when it is invoked the first
time. As we do not want to time the Julia \gls{JIT} compilation, we
need to force the system to \gls{JIT} compile the function before we
time the function. In Julia, this is done by simply calling the
function once. This is done prior to timing.

\begin{figure}[H]
  \begin{minted}[linenos,numbersep=5pt,gobble=2,frame=lines,framesep=2mm]{julia}
  foo()
  # ...
  @time foo()
  \end{minted}
  \caption{Triggering JIT before timing}
\end{figure}

### Timing Asynchronous Dispatch ###

Both the \gls{CUDA} and \gls{OpenCL} runtime libraries, provides
functions for moving data to and from the GPU and executing the
kernels on the \gls{GPU}. These functions, except copy from GPU to
CPU, are implemented as asynchronous calls, so that the GPU and CPU
can work concurrently. This means that timing an _execute kernel_ call
alone, will not actually time the execution of the kernel. To ensure
that we get the execution time, we insert a synchronization point
after the function and time these two together. Figure
\ref{fig:disc:async-call:sched} show how to time the scheduling of
execution, while Figure \ref{fig:disc:async-call:exec} times the
actual execution.

\newminted{julia}{linenos}
\begin{figure}[H]
  \begin{minted}[linenos,numbersep=5pt,gobble=2,frame=lines,framesep=2mm]{julia}
  @time Driver.execute(kernel)
  \end{minted}
  \caption{Timing Schedule Kernel}
  \label{fig:disc:async-call:sched}
\end{figure}

\begin{figure}[H]
  \begin{minted}[linenos,numbersep=5pt,gobble=2,frame=lines,framesep=2mm]{julia}
  @time begin
    Driver.execute(kernel)
    Driver.synchronize()
  end
  \end{minted}
  \caption{Timing Execution Time}
  \label{fig:disc:async-call:exec}
\end{figure}


## Programmer Productivity ##

This section discusses the results presented in Section
\ref{sec:res:productivity} of programmer productivity obtained by
PTX.jl compared to CUDA and OpenCL C.

### Lines of Codes ###

Table \ref{tab:loc} in Section \ref{sec:res:loc} shows that
the LOC metric is not decreased by using the PTX.jl library.The target
of the version of PTX.jl presented in this report is a fully working
system. It does not include a lot of the high level features of the
language. Therefore the poor performance on this metric was expected.

### Dynamic Typing ###

The fact that one kernel definition can handle multiple primitive
types, leads to a big potential reduction in LOC. A note here is that
CUDA has a C++ frontend and OpenCL might get one in the
future. Written in C++, a kernel can be templetized over multiple
types and implemented, providing the same reduction in LOC.

### Host and Device code in one language ###

The design philosophy of the Julia programming language is to enable
users of high level languages to take low level control within the
language. The main contribution of the PTX.jl library is to enable
this for GPU computing. This improves over the existing libraries
available for the language, where the programmer must use C or C++ to
define the kernels executed on the device.

\end{markdown}
